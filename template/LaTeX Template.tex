\documentclass{article} %Documentclass


% Add required usepackages to the project
\usepackage[utf8]{inputenc}
\usepackage[UKenglish]{babel} %\usepackage[portuguese]{babel} for portuguese language documents
\usepackage[table,xcdraw]{xcolor}
\usepackage{multicol}  
\usepackage{eurosym} 
\usepackage{geometry}
\usepackage{graphicx}   
\usepackage{float} 
\usepackage{wrapfig}
\usepackage{pdfpages}
\usepackage{amsmath} 
\usepackage{multirow}
\usepackage{indentfirst}
\usepackage{fancyhdr}
\usepackage{longtable}
\usepackage{lscape}
\usepackage{hyperref}
\usepackage{verbatim}
\usepackage{soul}
\usepackage{enumitem,amssymb}





\title{\LARGE\textbf{ADD TITLE}} 
\date{\today} %You can add the date manually
\author{ADD AUTHOR}


\begin{document} %Document stats here!


\begin{titlepage} % Create a different page for the Cover Page

\pagenumbering{gobble} %Remove page numbering


\begin{figure}[H]
\center 
\includegraphics[scale=0.5]{latex.png} % Add a Picture for Cover Page
\end{figure}

{\let\newpage\relax\maketitle} 

\vspace*{150px} %Space between Figure and Footer

\begin{small}
\begin{center}
Write here what you want to appear at the bottom of the cover.
\end{center}
\end{small}


\end{titlepage} % Close the cover page
\pagenumbering{arabic} % The page numbering starts from here. If it is not of interest to have numbering, just comment the function :)



\tableofcontents % Index
\listoffigures % Image Index
\listoftables % Table Index

\newpage 
\section{Introduction}
    \subsection{What is \LaTeX?}
    \quad \LaTeX (usually pronounced ``LAY teck,'' sometimes ``LAH teck,'' and never ``LAY tex'') is a mathematics typesetting program that is the standard for most professional mathematics writing. It is based on the typesetting program \TeX\ created by Donald Knuth of Stanford University (his first version appeared in 1978). Leslie Lamport was responsible for creating \LaTeX\, a more user friendly version of \TeX. A team of \LaTeX\ programmers created the current version,  \LaTeX\ 2$\varepsilon$.
    
    
    \subsection{What can I do with it?}
           \begin{multicols}{2}
        \begin{itemize}
            \item Reports;
            \item Thesis;
            \item Papers; 
            \item Dissertations;
            \item Presentations;
            \item CV's;
            \item Formal Letters;
            \item Newsletters;
            \item Posters. 
        \end{itemize} 
        \end{multicols}
    
        \quad % better formatting in each paragraph.
    \subsection{How does it work?}
    \begin{figure}[H]
        \centering
        \includegraphics[]{1.png}
        \caption{How does it work \LaTeX - 1}
        \label{fig:my_label}
    \end{figure}
        \begin{figure}[H]
        \centering
        \includegraphics[]{2.png}
        \caption{How does it work \LaTeX - 2}
        \label{fig:my_label}
    \end{figure}
    
    \subsection{Structure}
    \begin{multicols}{2}
    \begin{itemize}
        \item Preamble
        \begin{itemize}
            \item Define type of document;
            \item Packages to use;
            \item title, author(s), date, etc;
        \end{itemize}
        \item Body
        \begin{itemize}
            \item Actual content of the document
            \item sections, subsections, etc;
            \item lists, images, tables, equations, etc;
        \end{itemize}
    \end{itemize}
    \end{multicols}
    
    \subsection{Text decorations}
    \begin{itemize}
        \item Your text can be:
            \begin{itemize}
                \item \textit{italics} (\verb!\textit{italics}!);
                \item \textbf{boldface} (\verb!\textbf{boldface}!);
                \item \underline{underlined} (\verb!\underline{underlined}!).
            \end{itemize}
        
        \item Your math can contain boldface:
            \begin{multicols}{2}
            \begin{itemize}
                \item $\mathbf{R}$ (\verb!\mathbf{R}!);
                \item blackboard bold, $\mathbb{R}$ (\verb!\mathbb{R}!).
            \end{itemize}
            \end{multicols}
       
        \item You may want to used these to express the sets of:
            \begin{multicols}{2}
            \begin{itemize}
                \item real numbers ($\mathbb{R}$ or $\mathbf{R}$);
                \item integers ($\mathbb{Z}$ or $\mathbf{Z}$);
                \item rational numbers ($\mathbb{Q}$ or $\mathbf{Q}$);
                \item and natural numbers ($\mathbb{N}$ or $\mathbf{N}$).
            \end{itemize}
            \end{multicols}
        
        \item To have text appear in a math expression use \verb!\text!.
        %\verb!(0,1]=\{x\in\mathbb{R}:x>0 \text{ and }x\le 1\}! yields $(0,1]=\{x\in\mathbb{R}:x>0\text{ and }x\le 1\}$. (Without the \verb!\text! command it treats ``and'' as three variables: $(0,1]=\{x\in\mathbb{R}:x>0 and x\le 1\}$.)
    \end{itemize}
            

    \subsection{Other useful tips}
    \subsubsection{Spaces and new lines}

        \quad \LaTeX\ ignores extra spaces and new lines. For example:
        
        \verb!This   sentence will       look!
        \verb!fine after      it is     compiled.!

        \quad This   sentence will       look fine after      it is     compiled.


        \quad Leave one full empty line between two paragraphs. Place \verb!\\! at the end of a line to create a new line (but not create a new paragraph).

        \verb!This!
        
        \verb!compiles!
        
        ~
        
        \verb!like\\!
        
        \verb!this.!
        
        This
        compiles 
        
        like\\
        this.
        
        Use  \verb!\noindent! to prevent a paragraph from indenting.
        
    \subsubsection{Comments}
        
        \quad Use \verb!%! to create a comment. Nothing on the line after the: \\ \verb!%! will be typeset. \verb!$f(x)=\sin(x)$ %this is the sine function! yields $f(x)=\sin(x)$ %this is the sine function

    \subsubsection{Delimiters}

        \begin{center}
        \begin{tabular}{l|l|l}
        \emph{description} & \emph{command} & \emph{output}\\ \hline
        parentheses &\verb!(x)! & (x)\\
        brackets &\verb![x]! & [x]\\
        curly braces& \verb!\{x\}! & \{x\}\\
        \end{tabular}
        \end{center}
        
        \vspace{0.5cm}
        \quad To make your delimiters large enough to fit the content, use them together with \verb!\right! and \verb!\left!. For example: \verb!\left\{\sin\left(\frac{1}{n}\right)\right\}_{n}^! \verb!{\infty}! produces\\ $\displaystyle \left\{\sin\left(\frac{1}{n}\right)\right\}_{n}^{\infty}$.
        
        \vspace{0.5cm}
        \quad Curly braces are non-printing characters that are used to gather text that has more than one character. Observe the differences between the four expressions \verb!x^2!, \verb!x^{2}!, \verb!x^2t!, \verb!x^{2t}! when typeset: $x^2$, $x^{2}$, $x^2t$, $x^{2t}$.








  
\newpage
\section{Body}
    \subsection{Add Figures}
    \quad You can put images (pdf, png, jpg, or gif) in your document. They need to be in the same location as your .tex file when you compile the document. Omit   \verb![width=.5in]! if you want the image to be full-sized.
        
        \begin{center}
        \verb!\begin{figure}[ht]! \\
        \verb!\includegraphics[width=.5in]{imagename.jpg}! \\
        \verb!\caption{The (optional) caption goes here.}! \\
        \verb!\end{figure}!            
        \end{center}

%\noindent\makebox[\linewidth]{\rule{\paperwidth}{0.4pt}}
\noindent\rule{16cm}{0.4pt}

        \begin{figure}[H]
            \centering % \centering or \center
            \includegraphics[scale=0.4]{latex.png} % scale the size.
            \caption{\LaTeX Logo}
            \label{fig:\LaTeX}
        \end{figure}
    
    \subsection{ Add Tables}
% "|" make the dividing line 
% Each "&" is a cell separator and the double-backslash "\\" sets the end of this row.

    \begin{table}[H]
    \center
        \begin{tabular}{|c|c c c|}
            \toprule
            \textbf{x} & \textbf{1} & \textbf{2} & \textbf{3} \\ \midrule \hline
            \textbf{1} & 1          & 2          & 3          \\ 
            \textbf{2} & 2          & 4          & 6          \\ 
            \textbf{3} & 3          & 6          & 9          \\ \bottomrule
        \end{tabular}
    \end{table}

    \newline
    
    \quad For more complex tables it is recommended to use the following websites:
    \begin{itemize}
        \item \href{https://tableconvert.com/?output=latex }{Table Convert}
        \item \href{https://www.tablesgenerator.com/ }{Table Generator}
        \item \href{https://github.com/krlmlr/Excel2LaTeX}{Excel2LaTeX} - Add-in for excel
    \end{itemize}
    
    
    
\subsection{Lists}

\quad You can produce ordered and unordered lists.

    \begin{center}
    \begin{tabular}{l|l|l}
    \emph{description} & \emph{command} & \emph{output}\\ \hline
    unordered list&
    \begin{tabular}{l}
    \verb!\begin{itemize}!\\
    \verb!  \item!\\
    \verb!  Thing 1!\\
    \verb!  \item!\\
    \verb!  Thing 2!\\
    \verb!\end{itemize}!
    \end{tabular}&
    \begin{tabular}{l}
    $\bullet$ Thing 1\\
    $\bullet$ Thing 2
    \end{tabular}\\ \\ \hline
    ordered list&
    \begin{tabular}{l}
    \verb!\begin{enumerate}!\\
    \verb!  \item!\\
    \verb!  Thing 1!\\
    \verb!  \item!\\
    \verb!  Thing 2!\\
    \verb!\end{enumerate}!
    \end{tabular}&
    \begin{tabular}{l}
    1.~Thing 1\\
    2.~Thing 2
    \end{tabular}
    \end{tabular}
    \end{center}

\newpage
\section{Math vs. text vs.Functions}
    \quad In properly typeset mathematics  variables appear in italics (e.g., $f(x)=x^{2}+2x-3$). The exception to this rule is predefined functions (e.g., $\sin (x)$). Thus it is important to \textbf{always} treat text, variables, and functions correctly. See the difference between $x$ and x, -1 and $-1$, and $sin(x)$ and $\sin(x)$. There are two ways to present a mathematical expression--- \emph{inline} or as an \emph{equation}.

    \subsection{Inline mathematical expressions}
    \quad Inline expressions occur in the middle of a sentence.  To produce an inline expression, place the math expression between dollar signs (\verb!$!).  For example:
    
    \quad typing \verb!$90^{\circ}$ is the same as $\frac{\pi}{2}$radians! $90^{\circ}$ is the same as $\frac{\pi}{2}$ radians.

    \subsection{Equations}
    \quad Equations are mathematical expressions that are given their own line and are centered on the page.  These are usually used for important equations that deserve to be showcased on their own line or for large equations that cannot fit inline. To produce an inline expression, place the mathematical expression  between the symbols  \verb!\[! and \verb!\]!. 
    
    Typing \verb!\[x=\frac{-b\pm\sqrt{b^2-4ac}}{2a}\]! yields \[x=\frac{-b\pm\sqrt{b^2-4ac}}{2a}.\]
    
    Or you can just start a new "equation":
    \begin{equation}
        x= \frac{-b\pm\sqrt{b^2-4ac}}{2a}
    \end{equation}
 
    \subsection{Displaystyle} 
    \quad I want  this $\displaystyle \sum_{n=1}^{\infty}\frac{1}{n}$, not this $\sum_{n=1}^{\infty}\frac{1}{n}.$

\newpage
\section{Symbols}

    \subsection{Symbols (in \emph{text} mode)}

    \quad The followign symbols do \textbf{not} have to be surrounded by dollar signs.

\newline

        \begin{center}
        \begin{tabular}{l|l|l}
        \emph{description} & \emph{command} & \emph{output}\\ \hline
        dollar sign & \verb!\$! & \$ \\
        percent & \verb!\%! & \% \\
        ampersand & \verb!\&! & \& \\
        pound & \verb!\#! & \# \\
        backslash & \verb!\textbackslash! & \textbackslash \\
        left quote marks & \verb!``! & `` \\
        right quote marks & \verb!''! & '' \\
        single left quote  & \verb!`! & ` \\
        single right quote  & \verb!'! & ' \\
        hyphen & \verb!X-ray! & X-ray\\
        en-dash & \verb!pp. 5--15! & pp. 5--15 \\
        em-dash & \verb!Yes---or no?! & Yes---or no? 
        \end{tabular}
        \end{center}

\subsection{Symbols (in \emph{math} mode)}
    \subsubsection{The basics}
        \begin{center}
        \begin{tabular}{l|l|l}
        \emph{description} & \emph{command} & \emph{output}\\ \hline
        addition & \verb!+! & $+$\\
        subtraction & \verb!-! & $-$\\
        plus or minus & \verb!\pm! & $\pm$\\
        multiplication (times) & \verb!\times! & $\times$\\
        multiplication (dot) & \verb!\cdot! & $\cdot$\\
        division symbol & \verb!\div! & $\div$\\
        division (slash) & \verb!/! & $/$\\
        circle plus & \verb!\oplus! & $\oplus$\\
        circle times & \verb!\otimes! & $\otimes$\\
        equal & \verb!=! & $=$\\
        not equal & \verb!\ne! & $\ne$\\
        less than & \verb!<! & $<$\\
        greater than & \verb!>! & $>$\\
        less than or equal to & \verb!\le! & $\le$\\
        greater than or equal to & \verb!\ge! & $\ge$\\
        approximately equal to & \verb!\approx! & $\approx$\\
        infinity & \verb!\infty! & $\infty$\\
        dots & \verb!1,2,3,\ldots! & $1,2,3,\ldots$\\
        dots & \verb!1+2+3+\cdots! & $1+2+3+\cdots$\\
        fraction & \verb!\frac{a}{b}! & $\frac{a}{b}$\\
        square root & \verb!\sqrt{x}! & $\sqrt{x}$\\
        $n$th root & \verb!\sqrt[n]{x}! & $\sqrt[n]{x}$\\
        exponentiation & \verb!a^b! & $a^{b}$\\
        subscript & \verb!a_b! & $a_{b}$\\
        absolute value & \verb!|x|! & $|x|$\\
        natural log  & \verb!\ln(x)! & $\ln(x)$\\
        logarithms & \verb!\log_{a}b! & $\log_{a}b$\\
        exponential function & \verb!e^x=\exp(x)! & $e^{x}=\exp(x)$\\
        degree & \verb!\deg(f)! & $\deg(f)$\\
        \end{tabular}
        \end{center}
    \newpage

    \subsubsection{Functions}
        \begin{center}
        \begin{tabular}{l|l|l}
        \emph{description} & \emph{command} & \emph{output}\\ \hline
        maps to & \verb!\to! & $\to$\\
        composition& \verb!\circ! & $\circ$\\
        piecewise& \verb!|x|=! & \multirow{5}{*}{$\displaystyle |x|=\begin{cases}x&x\ge 0\\-x&x<0\end{cases}$}\\
        function&\verb!\begin{cases}!&\\ 
        &\verb!x & x\ge 0\\!&\\ 
        &\verb!-x & x<0!&\\ 
        &\verb!\end{cases}!&
        \end{tabular}
        \end{center}

    \subsubsection{Greek and Hebrew letters}
        \begin{center}
        \begin{tabular}{l|l|l|l}
        \emph{command} & \emph{output}&\emph{command} & \emph{output}\\ \hline
        \verb!\alpha! & $\alpha$&\verb!\tau! & $\tau$\\
        \verb!\beta! & $\beta$&\verb!\theta! & $\theta$\\
        \verb!\chi! & $\chi$&\verb!\upsilon! & $\upsilon$\\
        \verb!\delta! & $\delta$&\verb!\xi! & $\xi$\\
        \verb!\epsilon! & $\epsilon$&\verb!\zeta! & $\zeta$\\
        \verb!\varepsilon! & $\varepsilon$&\verb!\Delta! & $\Delta$\\
        \verb!\eta! & $\eta$&\verb!\Gamma! & $\Gamma$\\
        \verb!\gamma! & $\gamma$&\verb!\Lambda! & $\Lambda$\\
        \verb!\iota! & $\iota$&\verb!\Omega! & $\Omega$\\
        \verb!\kappa! & $\kappa$&\verb!\Phi! & $\Phi$\\
        \verb!\lambda! & $\lambda$&\verb!\Pi! & $\Pi$\\
        \verb!\mu! & $\mu$&\verb!\Psi! & $\Psi$\\
        \verb!\nu! & $\nu$&\verb!\Sigma! & $\Sigma$\\
        \verb!\omega! & $\omega$&\verb!\Theta! & $\Theta$\\
        \verb!\phi! & $\phi$&\verb!\Upsilon! & $\Upsilon$\\
        \verb!\varphi! & $\varphi$&\verb!\Xi! & $\Xi$\\
        \verb!\pi! & $\pi$&\verb!\aleph! & $\aleph$\\
        \verb!\psi! & $\psi$&\verb!\beth! & $\beth$\\
        \verb!\rho! & $\rho$&\verb!\daleth! & $\daleth$\\
        \verb!\sigma! & $\sigma$&\verb!\gimel! & $\gimel$
        \end{tabular}
        \end{center}

    \subsubsection{Set theory}
        \begin{center}
        \begin{tabular}{l|l|l}
        \emph{description} & \emph{command} & \emph{output}\\ \hline
        set brackets & \verb!\{1,2,3\}! & $\{1,2,3\}$\\
        element of & \verb!\in! & $\in$\\
        not an element of & \verb!\not\in! & $\not\in$\\
        subset of & \verb!\subset! & $\subset$\\
        subset of & \verb!\subseteq! & $\subseteq$\\
        not a subset of & \verb!\not\subset! & $\not\subset$\\
        contains & \verb!\supset! & $\supset$\\
        contains & \verb!\supseteq! & $\supseteq$\\
        union & \verb!\cup! & $\cup$\\
        intersection & \verb!\cap! & $\cap$\\
        big union & 
        \verb!\bigcup_{n=1}^{10}A_n! &
        $\displaystyle \bigcup_{n=1}^{10}A_{n}$\\
        big intersection & \verb!\bigcap_{n=1}^{10}A_n! &$\displaystyle \bigcap_{n=1}^{10}A_{n}$\\
        empty set & \verb!\emptyset! & $\emptyset$\\
        power set & \verb!\mathcal{P}! & $\mathcal{P}$\\
        minimum & \verb!\min! & $\min$\\
        maximum & \verb!\max! & $\max$\\
        supremum & \verb!\sup! & $\sup$\\
        infimum & \verb!\inf! & $\inf$\\
        limit superior & \verb!\limsup! & $\limsup$\\
        limit inferior & \verb!\liminf! & $\liminf$\\
        closure & \verb!\overline{A}! & $\overline{A}$
        \end{tabular}
        \end{center}

    \subsubsection{Calculus}
        \begin{center}
        \begin{tabular}{l|l|l}
        \emph{description} & \emph{command} & \emph{output}\\ \hline
        derivative & \verb!\frac{df}{dx}! & $\displaystyle \frac{df}{dx}$\\
        derivative & \verb!\f'! & $f'$\\
        partial derivative & 
        \begin{tabular}{l}
        \verb!\frac{\partial f}!\\ \verb!{\partial x}! 
        \end{tabular}& $\displaystyle \frac{\partial f}{\partial x}$\\
        integral & \verb!\int! & $\displaystyle\int$\\
        double integral & \verb!\iint! & $\displaystyle\iint$\\
        triple integral & \verb!\iiint! & $\displaystyle\iiint$\\
        limits & \verb!\lim_{x\to \infty}! & $\displaystyle \lim_{x\to \infty}$\\
        summation  & 
        \verb!\sum_{n=1}^{\infty}a_n! &
        $\displaystyle \sum_{n=1}^{\infty}a_n$\\
        product  & 
        \verb!\prod_{n=1}^{\infty}a_n! &
        $\displaystyle \prod_{n=1}^{\infty}a_n$
        \end{tabular}
        \end{center}

    \subsubsection{Logic}
        \begin{center}
        \begin{tabular}{l|l|l}
        \emph{description} & \emph{command} & \emph{output}\\ \hline
        not & \verb!\sim! & $\sim$\\
        and & \verb!\land! & $\land$\\
        or & \verb!\lor! & $\lor$\\
        if...then & \verb!\to! & $\to$\\
        if and only if & \verb!\leftrightarrow! & $\leftrightarrow$\\
        logical equivalence & \verb!\equiv! & $\equiv$\\
        therefore & \verb!\therefore! & $\therefore$\\
        there exists  & \verb!\exists! & $\exists$\\
        for all & \verb!\forall! & $\forall$\\
        implies & \verb!\Rightarrow! & $\Rightarrow$\\
        equivalent & \verb!\Leftrightarrow! & $\Leftrightarrow$
        \end{tabular}
        \end{center}

    \subsubsection{Linear algebra}
        \begin{center}
        \begin{tabular}{l|l|l}
        \emph{description} & \emph{command} & \emph{output}\\ \hline
        vector & \verb!\vec{v}! & $\vec{v}$\\
        vector & \verb!\mathbf{v}! & $\mathbf{v}$\\
        norm & \verb!||\vec{v}||! & $||\vec{v}||$\\
        matrix&
        \begin{tabular}{l}
        \verb!\left[!\\
        \verb!\begin{array}{ccc}!\\
        \verb!1 & 2 & 3 \\!\\
        \verb!4 & 5 & 6\\!\\
        \verb!7 & 8 & 0!\\
        \verb!\end{array}!\\
        \verb!\right]!\end{tabular}&
        $\displaystyle \left[\begin{array}{ccc}1 & 2 & 3 \\4 & 5 & 6 \\7 & 8 & 0\end{array}\right]$\\
        \\determinant&
        \begin{tabular}{l}
        \verb!\left|!\\
        \verb!\begin{array}{ccc}!\\
        \verb!1 & 2 & 3 \\!\\
        \verb!4 & 5 & 6 \\!\\
        \verb!7 & 8 & 0!\\
        \verb!\end{array}!\\
        \verb!\right|!
        \end{tabular}&
        $\displaystyle \left|\begin{array}{ccc}1 & 2 & 3 \\4 & 5 & 6 \\7 & 8 & 0\end{array}\right|$\\
        determinant & \verb!\det(A)! & $ \det(A)$\\
        trace & \verb!\operatorname{tr}(A)! & $\operatorname{tr}(A)$\\
        dimension & \verb!\dim(V)! & $\dim(V)$\\
        \end{tabular}
        \end{center}

    \subsubsection{Number theory}
        \begin{center}
        \begin{tabular}{l|l|l}
        \emph{description} & \emph{command} & \emph{output}\\ \hline
        divides & \verb!|! & $|$\\
        does not divide & \verb!\not |! & $\not |$\\
        div & \verb!\operatorname{div}! & $\operatorname{div}$\\
        mod & \verb!\mod! & $\operatorname{mod}$\\
        greatest common divisor & \verb!\gcd! & $\gcd$\\
        ceiling & \verb!\lceil x \rceil! & $\lceil x\rceil$\\
        floor & \verb!\lfloor x \rfloor! & $\lfloor x \rfloor$\\
        \end{tabular}
        \end{center}


    \subsubsection{Geometry and trigonometry}
        \begin{center}
        \begin{tabular}{l|l|l}
        \emph{description} & \emph{command} & \emph{output}\\ \hline
        angle& \verb!\angle ABC! & $\angle ABC$\\
        degree& \verb!90^{\circ}! & $90^{\circ}$\\
        triangle& \verb!\triangle ABC! & $\triangle ABC$\\
        segment& \verb!\overline{AB}! & $\overline{AB}$\\
        sine& \verb!\sin! & $\sin$\\
        cosine& \verb!\cos! & $\cos$\\
        tangent& \verb!\tan! & $\tan$\\
        cotangent& \verb!\cot! & $\cot$\\
        secant& \verb!\sec! & $\sec$\\
        cosecant& \verb!\csc! & $\csc$\\
        inverse sine& \verb!\arcsin! & $\arcsin$\\
        inverse cosine& \verb!\arccos! & $\arccos$\\
        inverse tangent& \verb!\arctan! & $\arctan$\\
        \end{tabular}
        \end{center}


\end{document}